\documentclass{beamer}
\usetheme{AnnArbor}
\usecolortheme{spruce}

\usepackage [T2A] {fontenc}   % Кириллица в PDF файле
\usepackage [utf8] {inputenc} % Кодировка текста: utf-8
\usepackage [russian] {babel} % Переносы, лигатуры

\usepackage{fancyvrb}
\usepackage{tikz}
\usepackage{float}
\usepackage{stmaryrd}
\usepackage{mathrsfs}

\usetikzlibrary{arrows}

\newcommand{\tbttr}{T\_tr}
\newcommand{\tcoq}{Coq}
\newcommand{\tqc}{QuickChick}

\setbeamerfont{institute}{size=\normalsize}

\title[ ]{Верификация структуры данных <<зиппер>>}
\author[Грахов П.]{Грахов Павел, ПМИ}
\institute[ИВЭ ЮФУ]{Южный федеральный университет\\
Кафедра информатики и вычислительной техники\\
\vspace{0.5cm}
Научный руководитель --- ст.~преп. В.Н.~Брагилевский}
\date[2019]{Ростов-на-Дону\\2019}

\begin{document}

\frame{\titlepage}

% Постановка задачи
\begin{frame}
\frametitle{Постановка задачи}
\begin{itemize}
\item Формальная верификация алгоритмов (Coq)
\item Разработка надежного ПО
\item Тестирование алгоритмов и ПО
\item Алгоритмы $\rightarrow$ программы

\vspace{0.5cm}
\item {\bf Задача:} \\ Разработать для некоторой структуры данных Coq-проект, содержащий:
\begin{itemize}
\item Формализацию исходной структуры данных
\item Доказательство ее корректности
\item Тестирование
\item Генератор кода
\end{itemize}
В качестве исходной структуры выбран <<зиппер>> для дерева
\end{itemize}
\end{frame}

%Зиппер
\begin{frame}
\frametitle{Формализация <<зиппера>>}
\begin{itemize}
\item Пара курсор-контекст $(Z_T, Z_C)$
\item Введенные операции
\begin{itemize}
\item \textbf{MoveTop\ Z}
\item \textbf{MoveDown\ d\ Z}
\item \textbf{ZipperToTree\ Z}
\item \textbf{TreeToZipper\ T}
\item \textbf{Modify\ Z\ f}
\end{itemize}
\end{itemize}
\end{frame}

%Теоремы
\begin{frame}
\frametitle{Теоремы о <<зиппере>>}
\begin{itemize}
\item Операции над зиппером сохраняют свойства исходного дерева
\item Модификация курсора не меняет контекст
\item Функции навигации не меняют исходное дерево
\end{itemize}
\end{frame}

%Coq
\begin{frame}
\frametitle{Coq-формализация}
\begin{itemize}
\item Обработка ошибок усложняет доказательства
\item Использование автоматизации
\end{itemize}
\end{frame}

\begin{frame}[fragile]
\frametitle{Обработка ошибок}
\begin{itemize}
\item Проверяющий предикат
\begin{verbatim}
Definition CorrectMoveDownConditions D Z := ...
\end{verbatim}
\item Разные версии функции
\begin{verbatim}
Definition MoveDown D Z : ZipperTree := ...
Definition CheckAndMoveDown D Z :
    option ZipperTree := ...
Lemma SomethingAboutMoveDown:
    CorrectMoveDownConditions D Z -> ...
\end{verbatim}
\end{itemize}
\end{frame}

\begin{frame}
\frametitle{Автоматизация}
\begin{itemize}
\item Использование специфических тактик (как \texttt{omega} либо \texttt{ring})
\item Механизм подсказок (\texttt{Hint})
\item Изменение конфигурации ядра (\texttt{Add Relation}, \texttt{Transparent})
\end{itemize}
\end{frame}

\begin{frame}[fragile]
\frametitle{Тестирование}
\begin{itemize}
\item Возможно, теорема неверна?
\item Тестирование на большом количестве входных данных: QuickChick
\end{itemize}
\begin{verbatim}
Definition genInput : G (nat * (list nat)) :=
  genPair (choose (0, 100)) (listOf (choose (0, 100))).
Definition qc_test_insert_prop (nl: prod nat (list nat)) :=
  nth_insert_prop (fst nl) (snd nl).
QuickChick (forAll genInput qc_test_insert_prop).
\end{verbatim}
\end{frame}

\begin{frame}
\frametitle{Генерация кода}
\begin{itemize}
\item Выходной язык: Haskell
\item Прямая трансляция из Coq-кода будет работать \textbf{очень медленно}
\item Необходимы оптимизации (возможно, в ущерб формальной корректности)
\end{itemize}
\end{frame}

\begin{frame}[fragile]
\frametitle{Оптимизация генерации кода}
\begin{itemize}
\item Использование стандартных типов данных Haskell (\texttt{Integer})
\item \texttt{Inline} для простых определений
\end{itemize}
\end{frame}

% Полученные результаты
\begin{frame}
\frametitle{Полученные результаты}
\begin{itemize}
\item Доказана корректность <<зиппер>> для древовидных структур
\item Построен верифицированный генератор Haskell-кода для <<зиппера>>
\end{itemize}
\end{frame}

\end{document}

%%% Local Variables:
%%% mode: latex
%%% TeX-master: t
%%% End:
