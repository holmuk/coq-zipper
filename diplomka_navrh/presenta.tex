\documentclass{beamer}
\usetheme{AnnArbor}
\usecolortheme{spruce}

\usepackage [T2A] {fontenc}   % Кириллица в PDF файле
\usepackage [utf8] {inputenc} % Кодировка текста: utf-8
\usepackage [russian] {babel} % Переносы, лигатуры

\usepackage{fancyvrb}
\usepackage{tikz}
\usepackage{float}
\usepackage{stmaryrd}
\usepackage{mathrsfs}

\usetikzlibrary{arrows, positioning}

\setbeamerfont{institute}{size=\normalsize}

\title[ ]{Формальная верификация структуры данных <<зиппер>>}
\author[Грахов П.]{Грахов Павел, ПМИ}
\institute[ИВЭ ЮФУ]{Южный федеральный университет\\
Кафедра информатики и вычислительного эксперимента\\
\vspace{0.5cm}
Научный руководитель --- ст.~преп. В.Н.~Брагилевский}
\date[2019]{Ростов-на-Дону\\2019}

\begin{document}

\frame{\titlepage}

% Постановка задачи
%\begin{frame}
%\frametitle{Постановка задачи}
%\begin{itemize}
%\item Формальная верификация алгоритмов (Coq)
%\item Разработка надежного ПО
%\item Тестирование алгоритмов и ПО
%\item Алгоритмы $\rightarrow$ программы
%
%\vspace{0.5cm}
%\item {\bf Задача:} \\ Разработать для некоторой структуры данных Coq-проект, содержащий:
%\begin{itemize}
%\item Формализацию исходной структуры данных
%\item Доказательство ее корректности
%\item Тестирование
%\item Генератор кода
%\end{itemize}
%В качестве исходной структуры выбран <<зиппер>> для дерева
%\end{itemize}
%\end{frame}

\begin{frame}
\frametitle{Постановка задачи}

\begin{figure}[H]
\centering
\tikzstyle{mnode} = [rectangle, draw, fill=blue!20, text centered, rounded corners]
\tikzstyle{module} = [rectangle, draw, fill=red!20, text centered, rounded corners]
\tikzstyle{darrow} = [draw, thick, <->, shorten >=0.1cm, shorten <=0.1cm]
\tikzstyle{marrow} = [draw, thick, ->, shorten >=0.1cm, shorten <=0.1cm]
\tikzstyle{pc} = [draw, thick, -, shorten >=0.1cm, shorten <=0.1cm]
\begin{tikzpicture}[node distance = 2cm, auto]
\node[mnode, fill=white] (huet) {\shortstack{Статья о <<Зиппер>>\\ \textbf{<<The Zipper>>} \\ \textit{G\'{e}rard Huet, INRIA, 1997}}};
\node[mnode, below left = of huet, text width = 1cm] (coq) {Coq};
\node[module, below right = of huet, text width = 1cm] (m) {?};
\node[module, below of = huet, node distance = 4cm] (h) {Haskell};

\draw[blue, dashed] (-5, -3)  rectangle (5, -2);
\draw[darrow] (m) -- (coq);
\draw[marrow] (huet) --  (h); 
\end{tikzpicture}
\end{figure}
\end{frame}

\begin{frame}
\frametitle{Общая схема}

\begin{figure}[H]
\centering
\tikzstyle{mnode} = [rectangle, draw, fill=blue!20, text centered, rounded corners]
\tikzstyle{module} = [rectangle, draw, fill=red!20, text centered, rounded corners]
\tikzstyle{darrow} = [draw, thick, <->, shorten >=0.1cm, shorten <=0.1cm]
\tikzstyle{marrow} = [draw, thick, ->, shorten >=0.1cm, shorten <=0.1cm]
\tikzstyle{pc} = [draw, thick, -, shorten >=0.1cm, shorten <=0.1cm]
\begin{tikzpicture}[node distance = 2cm, auto]
\node[mnode] (list) {*.v};
\node[mnode, right of = list, node distance = 3cm] (aux) {*.v};
\node[mnode, right of = aux, node distance = 3cm] (all) {*.v};
\node[mnode, below of = aux] (zip) {TreeZipper.v};
\node[module, right of = zip, node distance = 6cm] (qc) {QuickChick};

\node[text centered, below of = zip] (ex) {Extraction.v};
\node[module, below of = ex] (has) {Haskell};

\draw[darrow] (list) -- (aux);
\draw[darrow] (aux) -- (all);
\draw[marrow] (aux) -- (zip);
\draw[marrow] (all) -- (zip);
\draw[darrow] (zip) -- (qc);
\draw[pc] (zip) -- (ex);
\draw[marrow] (ex) -- (has);
\end{tikzpicture}
\end{figure}
\end{frame}

%Зиппер
\begin{frame}[fragile]
\frametitle{<<Зиппер>>}
\begin{figure}[H]
\centering

\tikzset{
  treenode/.style = {align=center, inner sep=0pt, text centered,
    font=\sffamily},
  arn_r/.style = {treenode, circle, black, draw=black, 
    text width=1.5em, very thick}% arbre rouge noir, noeud rouge
}

\begin{tikzpicture}[->,>=stealth',level/.style={sibling distance = 2cm/#1,
  level distance = 2cm}] 
\node [arn_r] {1}
    child{ node [arn_r] {2} 
            child{ node [arn_r] {4} 
            }
            child{ node [arn_r] {5}
            }                            
    }
    child{ node [arn_r] {3}
            child{ node [arn_r] {6} 
            }
		}
	child{ node [arn_r] {4}
		}
; 
\draw[red, dashed] (-3,-5) rectangle (-1, -1);
\end{tikzpicture}
\caption{Zipper $(Z_T, Z_C)$}
\end{figure}
\end{frame}

%Теоремы
\begin{frame}
\frametitle{<<Зиппер>>}
\begin{itemize}
\item Операции над зиппером сохраняют свойства исходного дерева
\item Модификация курсора не меняет контекст
\item Функции навигации не меняют исходное дерево
\end{itemize}
\end{frame}

\begin{frame}[fragile]
\frametitle{Обработка ошибок}

\begin{figure}[H]
\centering
\tikzstyle{mnode} = [rectangle, draw, fill=blue!20, text centered, rounded corners]
\tikzstyle{module} = [rectangle, draw, fill=red!20, text centered, rounded corners]
\tikzstyle{darrow} = [draw, thick, <->, shorten >=0.1cm, shorten <=0.1cm]
\tikzstyle{marrow} = [draw, thick, -latex, shorten >=0.1cm, shorten <= 0.1cm]
\tikzstyle{pc} = [draw, thick, -, shorten >=0.1cm, shorten <=0.1cm]
\begin{tikzpicture}[node distance = 1cm, auto]
\node[mnode] (cond) {CorrectMoveDownConditions};
\node[mnode, below of = cond] (coqfun) {MoveDown};
\node[mnode, below of = coqfun] (lemma) {Lemma};
\node[module, below of = lemma, node distance = 4cm] {\shortstack{Haskell\\ MoveDown}};

\node[mnode, fill = white!20, dashed, right of = coqfun, node distance = 6cm] (tmp) {CheckAndMoveDown};

\draw[blue, dashed] (-3,-2.5) rectangle (3, 0.5);

\draw[marrow, shorten >=-2.5cm] (tmp) -- (ex.north);
\draw[marrow, shorten <=1cm, dashed] (coqfun) -- (tmp);

\end{tikzpicture}
\end{figure}
\end{frame}


\begin{frame}
\frametitle{Автоматизация}

\begin{figure}[H]
\centering
\tikzstyle{mnode} = [rectangle, draw, fill=white!20, rounded corners]
\tikzstyle{module} = [rectangle, draw, fill=red!20, text centered, rounded corners]
\tikzstyle{darrow} = [draw, thick, <->, shorten >=0.1cm, shorten <=0.1cm]
\tikzstyle{marrow} = [draw, thick, ->, shorten >=0.1cm, shorten <=0.1cm]
\tikzstyle{pc} = [draw, thick, -, shorten >=0.1cm, shorten <=0.1cm]
\begin{tikzpicture}[node distance = 2cm, auto]
\node[mnode] (kernel) {\texttt{\shortstack{Hint Unfold ... \\Opaque mod}}};
\node[mnode, fill=blue!20, below of = kernel] (lemma) {\texttt{Lemma ...}};
\node[module, right of = lemma, node distance = 4cm] (m) {\shortstack{omega\\ ring}};

\node[above of = kernel, node distance = 0.8cm] {<<Очевидные>> свойства};
\node[above of = m, node distance = 0.8cm] {Специфичные модули};
\draw[marrow] (kernel) -- (lemma);
\draw[marrow] (m) -- (lemma);
\end{tikzpicture}
\end{figure}
\end{frame}

\begin{frame}[fragile]
\frametitle{Тестирование}

\begin{figure}[H]
\centering
\tikzstyle{mnode} = [rectangle, draw, fill=blue!20, text centered, rounded corners, text width=3cm]
\tikzstyle{module} = [rectangle, draw, fill=red!20, text centered, rounded corners]
\tikzstyle{darrow} = [draw, thick, <->, shorten >=0.1cm, shorten <=0.1cm]
\tikzstyle{marrow} = [draw, thick, ->, shorten >=0.1cm, shorten <= 0.1cm]
\tikzstyle{pc} = [draw, thick, -, shorten >=0.1cm, shorten <=0.1cm]
\begin{tikzpicture}[node distance = 2cm, auto]
\node[mnode] (l1) {\shortstack{Lemma\\ Proof ???}};
\node[mnode, below of = l1] (l2) {\shortstack{Lemma\\ Proof ???}};
\node[text centered, text width = 3cm, below of = l2] (l3) {\textbf{...}};
\node[mnode, below of = l3] (final) {\shortstack{Lemma\\ Proof !!!}};

\node[module, right of = l1, node distance = 4cm] (qc1) {QuickChick};
\node[module, right of = l2, node distance = 4cm] (qc2) {QuickChick};
\node[module, right of = l3, node distance = 4cm] (qc3) {QuickChick};

\node[draw, circle, left of = l1, node distance = 3cm] (f) {$f$};

\draw[darrow, dashed] (f) -- (l1);
\draw[marrow] (l1) -- (qc1);
\draw[marrow] (l2) -- (qc2);
\draw[marrow] (l3) -- (qc3);

\draw[marrow] (qc1) -- (l2);
\draw[marrow] (qc2) -- (l3);
\draw[marrow] (qc3) -- (final);

\end{tikzpicture}
\end{figure}
\end{frame}

\begin{frame}
\frametitle{Генерация кода}
\begin{itemize}
\item Так будет медленно
\begin{figure}[H]
\centering
\tikzstyle{mnode} = [rectangle, draw, fill=blue!20, text centered, rounded corners, text width=3cm]
\tikzstyle{module} = [rectangle, draw, fill=red!20, text centered, rounded corners]
\tikzstyle{darrow} = [draw, thick, <->, shorten >=0.1cm, shorten <=0.1cm]
\tikzstyle{marrow} = [draw, thick, ->, shorten >=0.1cm, shorten <= 0.1cm]
\tikzstyle{pc} = [draw, thick, -, shorten >=0.1cm, shorten <=0.1cm]
\begin{tikzpicture}[node distance = 4cm, auto]
\node[mnode] (l1) {Coq};
\node[mnode, fill=white!20, right of = l1] (l2) (ex) {Extract};
\node[module, right of = ex, fill=red] (h) {Haskell};

\draw[marrow] (l1) -- (ex);
\draw[marrow] (ex) -- (h);
\end{tikzpicture}
\end{figure}

\item Нужно контролировать процесс
\begin{figure}[H]
\centering
\tikzstyle{mnode} = [rectangle, draw, fill=blue!20, text centered, rounded corners, text width=3cm]
\tikzstyle{module} = [rectangle, draw, fill=red!20, text centered, rounded corners]
\tikzstyle{darrow} = [draw, thick, <->, shorten >=0.1cm, shorten <=0.1cm]
\tikzstyle{marrow} = [draw, thick, ->, shorten >=0.1cm, shorten <= 0.1cm]
\tikzstyle{pc} = [draw, thick, -, shorten >=0.1cm, shorten <=0.1cm]
\begin{tikzpicture}[node distance = 4cm, auto]
\node[mnode] (l1) {Coq};
\node[mnode, fill=white!20, right of = l1] (l2) (ex) {\shortstack{$\mathbb{Z} \rightarrow \texttt{Integer}$\\ \texttt{Inline ...} \\ Extract}};
\node[module, right of = ex] (h) {Haskell};

\draw[marrow] (l1) -- (ex);
\draw[marrow] (ex) -- (h);
\end{tikzpicture}
\end{figure}
\end{itemize}
\end{frame}

% Полученные результаты
\begin{frame}
\frametitle{Полученные результаты}
\begin{itemize}
\item Доказана корректность <<зиппер>> для древовидных структур
\item Построен верифицированный генератор Haskell-кода для <<зиппера>>
\end{itemize}
\end{frame}

\end{document}

%%% Local Variables:
%%% mode: latex
%%% TeX-master: t
%%% End:
